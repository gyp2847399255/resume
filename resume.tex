% !TEX program = xelatex

\documentclass{resume}
%\usepackage{zh_CN-Adobefonts_external} % Simplified Chinese Support using external fonts (./fonts/zh_CN-Adobe/)
%\usepackage{zh_CN-Adobefonts_internal} % Simplified Chinese Support using system fonts

\begin{document}
\pagenumbering{gobble} % suppress displaying page number

\name{Guo Yanpei}

\basicInfo{
  \email{gyp2847399255@gmail.com} \textperiodcentered\ 
  \phone{(+86) 13693134669}
}

\section{\faGraduationCap\ Education}
\datedsubsection{\textbf{Beihang University (BUAA)}, Beijing, China}{2019 -- 2023}
\textit{B.S.} in School of Computer Science and Engineering, GPA: 3.78 / 4.00

\section{\faBook\ Publication}

\datedsubsection{\textbf{Fast RS-IOP Multivariate Polynomial Commitments and Verifiable Secret Sharing}}{}
Zongyang Zhang*, Weihan Li*, \textbf{Yanpei Guo*}, Kexin Shi, Sherman S. M. Chow, Ximeng Liu, Jin Dong

\textit{USENIX Security Symposium (Security), 2024}, Co-first Author.

\textbf{Achievements:}
\begin{itemize}
  \item Rolling-batch FRI technique, batch proving proximity of multiple vectors with Reed-Solomon codes of different length. 
  \textbf{This technique has been applied in open-source zero-knowledge project Plonky3}.
  \item PolyFRIM, a multivariate polynomial commitment from RS code, leveraging rolling batch FRI, 5 times faster than prior work.
  \item One-to-many proof for proving multiple evaluations to multiple verifiers based on PolyFRIM, accelerating proving by 4 times.
  \item An asynchronous verifiable secret sharing (AVSS) scheme FRISS with a dealer complexity of $O(n^2 \log n)$. 
\end{itemize}
\textbf{My Duty:} Protocol design and proof, experiments

\section{\faBookmark\ Manuscript}

\datedsubsection{\textbf{Succinct Hash-based Arbitrary-Range Proofs}}{}
Zongyang Zhang, Weihan Li, \textbf{Yanpei Guo}, Sherman S. M. Chow, Zhiguo Wan

Submitted to \textit{IEEE Transactions on Information Forensics and Security(TIFS)}

\textbf{Achievements:}
\begin{itemize}
  \item A plausible post-quantum secure range proof (RP).
  \item A general bit-composition framework for ZK-RPs.
\end{itemize}
\textbf{My Duty:} Experiments

\datedsubsection{\textbf{Doubly-Efficient Multilinear Polynomial Commitment from Reed-Solomon Code and Its Applications to Zero Knowledge Proof}}{}
\textbf{Yanpei Guo}, Kexi Huang, Tianyang Tao, Jiaheng Zhang

Submitted to \textit{S\&P, 2025}, First Author

\textbf{Achievements:}
\begin{itemize}
  \item Deepfold, a multilinear polynomial commitment scheme that achieves nearly identical prover time, verifier time, and proof size as DEEP-FRI.
  It has about 3.5 times faster prover time or 3 times smaller proof size than prior work.
  \item A batch variant of Deepfold, reducing the prover time by half while adding no additional overhead to the proof size.
  \item A high-performance zero-knowledge argument system, by combining Deepfold with Libra.
\end{itemize}
\textbf{My Duty:} Protocol design and proof, experiments, paper writing

\section{\faCogs\ Skills}
\begin{itemize}[parsep=0.5ex]
  \item Programming Languages: Rust, C++, Golang, Python, Vue, SQL
  \item Platform: Linux
  \item Development: Database, Back-end, Front-end
\end{itemize}

\section{\faTrophy\ Honors and Awards}
\datedsubsection{\textbf{National Cryptographic technology Competition}}{2023, \textit{First Prize}}

\datedsubsection{\textbf{Computer System Development Capability Competition, Operating System Kernel Design}}{2022,  \textit{First Prize, First Place}}

\textbf{Achievements:}
\begin{itemize}
  \item Developed a microkernel operating system in C, compatible with QEMU and SiFive FU740 platforms.
  \item Integrated support for Busybox, Lua, Redis, GCC, and Vim.
  \item Significantly enhanced the FAT32 file system with optimizations that improve performance by up to two times compared to Linux in certain benchmarks.
\end{itemize}

\datedsubsection{\textbf{International Collegiate Programming Contest (ICPC) Asia Regional Contest Jinan Site}}{2020, \textit{Silver Medal, the 54th Place}}

\datedsubsection{\textbf{The Chinese Mathematics Competitions for College Students (CMC)}}{2020, \textit{First Prize}}

\section{\faHeartO\ TA experience}
\datedsubsection{\textbf{Computer Organization}, Beijing, China}{2022.07 -- 2023.01}
Superviaor: Xiaopeng Gao, School of Computer Science and Engineering, Beihang University
\datedsubsection{\textbf{Operating System}, Beijing, China}{2021.01 -- 2022.06}
Superviaor: Lei Wang, School of Computer Science and Engineering, Beihang University
\datedsubsection{\textbf{Computer Organization}, Beijing, China}{2021.07 -- 2022.01}
Superviaor: Xiaopeng Gao, School of Computer Science and Engineering, Beihang University


\section{\faCertificate\ Language}
\datedsubsection{\textbf{TOEFL}: Reading 28, Listening 23, Speaking 22, Writing 27}{2024.04}


%% Reference
%\newpage
%\bibliographystyle{IEEETran}
%\bibliography{mycite}
\end{document}
